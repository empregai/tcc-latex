% ----------------------------------------------------------
\chapter{REFERENCIAL TEÓRICO}\label{cap:desenvolvimento}
% ----------------------------------------------------------

% ----------------------------------------------------------
\subsection{Modelo incremental}

% ----------------------------------------------------------
\textcite{Pressman2016} afirma que os modelos tradicionais de processos de desenvolvimento se concentram em estruturar e ordenar o desenvolvimento de 
software. Nesses modelos, as atividades e tarefas ocorrem sequencialmente, seguindo diretrizes de progresso bem definidas. O autor também define algumas 
atividades genéricas para os processos de desenvolvimento de software, que são::
\begin{itemize}[label=$\bullet$]
	\item Comunicação: Levantamento de requisitos.
	\item Planejamento: Estimativas, cronograma, acompanhamento.
	\item Modelagem: Análise, projeto.
	\item Construção: Código, testes.
	\item Entrega/Disponibilização: Entrega, feedback.
	\end{itemize}
	De acordo com \textcite{Pressman2016}, o modelo incremental combina os fluxos de processo linear e paralelo dos elementos. Esse modelo é aplicado por
	meio de sequências lineares escalonadas à medida que o tempo avança. Cada sequência linear produz incrementos entregáveis do software. \newline
	\textcite{Pressman2016} destaca que, ao utilizar um modelo incremental, o primeiro incremento frequentemente é um produto essencial, fundamental 
	para atender às necessidades iniciais do cliente. Esse produto essencial é utilizado pelo cliente ou passa por uma avaliação detalhada para obter feedback valioso. 
	Com base no uso e/ou na avaliação, é desenvolvido um planejamento para o próximo incremento, levando em consideração as modificações necessárias para melhor adequar o 
	produto às necessidades do cliente, bem como a entrega de recursos e funcionalidades adicionais.
	\newline Dessa forma, o projeto em questão tem como objetivo utilizar o processo incremental para o desenvolvimento do software Empregai, seguindo validações de progresso quinzenais.
	
\subsection{React Native}
O React Native é uma poderosa ferramenta de desenvolvimento que permite criar aplicativos móveis nativos para iOS e Android usando JavaScript e a biblioteca React. Essa abordagem eficiente e produtiva para a criação de aplicativos móveis multiplataforma tem impulsionado o crescimento do React Native no mercado.

Desenvolvido e mantido pela empresa META, também proprietária do Facebook, o framework React Native é altamente confiável, beneficiando-se de uma comunidade ativa 
que disponibiliza uma ampla gama de conteúdos gratuitos online. A tendência atual para novos aplicativos móveis é o desenvolvimento voltado principalmente para as 
plataformas Android e iOS. Com o React Native, é possível adotar uma abordagem híbrida, permitindo a construção simultânea de um produto para ambas as plataformas, 
evitando a necessidade de desenvolvimento separado usando as linguagens nativas de cada plataforma, conforme destacado por (\textcite{Sabino}).

Outro fator determinante na escolha do React Native como framework é a sua curva de aprendizado acessível. Devido à ampla familiaridade e popularidade da linguagem 
JavaScript no mundo do desenvolvimento, a adoção do React Native é evidente (Figura 1), destacando sua força e importância no mercado.
\begin{figure}[htb]
	\caption{\label{fig:Fig_1}Linguagem de programção}
	\begin{center}
		\includegraphics{images/top.png}
	\end{center}
	\fonte{https://www.softermii.com/blog/top-programming-languages-and-frameworks-for-software-development}
\end{figure}
% ----------------------------------------------------------

\subsection{Expo}
O Expo é uma plataforma que simplifica o desenvolvimento de aplicativos móveis usando JavaScript e React Native. Com recursos nativos do dispositivo acessíveis e facilitando a colaboração e distribuição de aplicativos, o Expo oferece uma solução abrangente para criar aplicativos móveis.

Segundo (\textcite{Hugo}) o uso do Expo proporciona uma camada de abstração superior ao React Native, resultando em uma experiência aprimorada no desenvolvimento de software. Com o aumento do 
número de usuários de smartphones, especialmente nas plataformas Android e iOS, a necessidade de criar aplicativos para ambas as plataformas se tornou cada vez mais 
evidente. Nesse contexto, o Expo oferece uma vantagem no desenvolvimento híbrido, simplificando o processo de criação de aplicativos multiplataforma.


\begin{figure}[htb]
	\caption{\label{fig:Fig_1}Expo vantagens}
	\begin{center}
		\includegraphics{images/expo.png}
	\end{center}
	\fonte{https://docs.expo.dev/core-concepts/}
\end{figure}

\subsection{Gin}
O framework Gin é uma biblioteca leve e rápida para construção de APIs em Go. Ele se baseia nos princípios do roteamento HTTP e fornece recursos poderosos para o desenvolvimento de aplicativos web escaláveis e de alto desempenho.

Ao utilizar o Gin, os desenvolvedores se beneficiam de uma sintaxe concisa e intuitiva, o que torna a criação de endpoints mais eficiente e produtiva. O framework oferece um roteamento flexível, permitindo mapear os diferentes endpoints para as funções correspondentes de forma clara e organizada.

Além disso, o Gin oferece suporte para a manipulação de parâmetros nas requisições HTTP. Isso permite que os desenvolvedores acessem e processem os dados enviados pelos clientes de forma simples e segura. O framework também oferece recursos para validação de entrada de dados, facilitando a verificação de formatos, tipos e restrições específicas, garantindo a integridade dos dados recebidos.

Outro aspecto importante do Gin é a sua eficiência e desempenho. Ele foi projetado para ser leve e rápido, proporcionando um processamento ágil das requisições. Isso é especialmente relevante em aplicações de grande escala, onde a capacidade de resposta e o tempo de processamento são cruciais.

O framework Gin também oferece recursos avançados, como middleware, que permite adicionar funcionalidades extras às rotas e endpoints da API. Isso inclui autenticação, autorização, logging e muitos outros aspectos que são essenciais para o desenvolvimento de sistemas seguros e escaláveis.

Em resumo, o uso do framework Gin proporciona uma base sólida e eficiente para o desenvolvimento de APIs RESTful. Sua sintaxe concisa, roteamento flexível, manipulação de parâmetros, validação de entrada e recursos avançados garantem a criação de endpoints robustos, escaláveis e de alto desempenho, possibilitando a construção de aplicações web modernas e eficientes.


\subsection{Comparação}
\begin{table}[htb]
    \centering
    \caption{Comparação}
    \label{tab:comparação}
\begin{tabular}{|p{5cm}|p{2cm}|p{2cm}|p{2cm}|}
    \hline
    \textbf{Características} & \textbf{Getninjas} & \textbf{99freelas} & \textbf{Empregai}  \\ \hline
    Diversidade de serviços   & \multicolumn{1}{c|}{\textbf{X}}   & \multicolumn{1}{c|}{\textbf{-}}  & \multicolumn{1}{c|}{\textbf{X}} \\ \hline
	Avaliações e recomendações   & \multicolumn{1}{c|}{\textbf{X}}   & \multicolumn{1}{c|}{\textbf{X}}  & \multicolumn{1}{c|}{\textbf{X}} \\ \hline
    Aplicativo móvel   & \multicolumn{1}{c|}{\textbf{X}}   & \multicolumn{1}{c|}{\textbf{-}}  & \multicolumn{1}{c|}{\textbf{X}} \\ \hline
    Sistema de mensagens   & \multicolumn{1}{c|}{\textbf{?}}   & \multicolumn{1}{c|}{\textbf{X}}  & \multicolumn{1}{c|}{\textbf{?}} \\ \hline
    Possibilidade dos clientes se adicionar   & \multicolumn{1}{c|}{\textbf{-}}   & \multicolumn{1}{c|}{\textbf{-}}  & \multicolumn{1}{c|}{\textbf{X}} \\ \hline
	Avaliação de pessoas conhecidas   & \multicolumn{1}{c|}{\textbf{-}}   & \multicolumn{1}{c|}{\textbf{-}}  & \multicolumn{1}{c|}{\textbf{X}} \\ \hline
	Pagamentos na plataforma   & \multicolumn{1}{c|}{\textbf{-}}   & \multicolumn{1}{c|}{\textbf{X}}  & \multicolumn{1}{c|}{\textbf{?}} \\ \hline
\end{tabular}
    \fonte{Elaborado pelos autores (2023).}
\end{table}