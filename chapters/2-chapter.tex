% ----------------------------------------------------------
\chapter{Desenvolvimento}\label{cap:desenvolvimento}
% ----------------------------------------------------------
Deve-se inserir texto entre as seções.
% ----------------------------------------------------------
\subsection{Levantamento de requisitos}
% ----------------------------------------------------------
\begin{quadro}[htb]
	\centering
	\caption{\label{Formatação do texto.}Requisitos funcionais}	
	\begin{tabular}{|l|p{11cm}|}
		\hline
		\textbf{Identificação}    & \textbf{Requisito}\\ \hline
		RF01        			  & Manter usuário\\ \hline
		RF02        			  & Realizar login\\ \hline
		RF03         			  & Cadastro de anúncio\\ \hline
		RF04        			  & Avaliação do profissional\\ \hline
		RF05        			  & Histórico de serviços prestados\\ \hline
		RF06        			  & Avaliação do profissional\\ \hline
	\end{tabular}
	\fonte{\textcite{Elaborado pelos autores(2023)}.}
\end{quadro}

\begin{quadro}[htb]
	\centering
	\caption{\label{Formatação do texto.}Requisitos não funcionais}	
	\begin{tabular}{|l|p{11cm}|}
		\hline
		\textbf{Identificação}    & \textbf{Requisito}\\ \hline
		RNF01        			  & Banco de dados relacional(PostgreSQL)\\ \hline
		RNF02        			  & Interface mobile em ReactJS Native\\ \hline
	\end{tabular}
	\fonte{\textcite{Elaborado pelos autores(2023)}.}
\end{quadro}

\begin{quadro}[htb]
	\centering
	\caption{\label{Formatação do texto.}Requisitos não funcionais}	
	\begin{tabular}{|l|p{11cm}|}
		\hline
		\textbf{Nome}    & Manter usuário\\ \hline
		\textbf{Tipo}    & Funcional\\ \hline
		\multicolumn{2}{|c|}{Descrição}\\ \hline
		\multicolumn{2}{|p{12cm}|}{
			O sistema deve permitir o cadastro de usuários e o gerenciamento de suas informações. \newline
			\newline Critérios de Aceitação: \newline
			O sistema deve permitir o cadastro de novos clientes com as informações básicas (nome, e-mail, senha, CPF, cidade, endereço e telefone); \newline
			\newline O sistema deve permitir o cadastro de novos prestadores com as informações básicas (nome, e-mail, senha, CPF, ou CNPJ, cidade, categoria do serviços prestados e telefone); \newline
			\newline O sistema deve permitir a atualização das informações do usuário, incluindo nome, e-mail e senha; \newline
			\newline O sistema deve permitir a exclusão do usuário; \newline
			O sistema deve garantir a segurança das informações dos usuários, armazenando as senhas de forma criptografada.
			} \\ \hline
	\end{tabular}
	\fonte{\textcite{Elaborado pelos autores(2023)}.}
\end{quadro}



% ----------------------------------------------------------
\subsubsection{As ilustrações}
% ----------------------------------------------------------

Independentemente do tipo de ilustração (quadro, desenho, figura, fotografia, mapa, entre outros), a sua identificação aparece na parte superior, precedida da palavra designativa. 

\begin{citacao}
	Após a ilustração, na parte inferior, indicar a fonte consultada (elemento obrigatório, mesmo que seja produção do próprio autor), legenda, notas e outras informações necessárias à sua compreensão (se houver). A ilustração deve ser citada no texto e inserida o mais próximo possível do texto a que se refere. \cite[p. 11]{NBR14724:2011}.
\end{citacao}

% ----------------------------------------------------------
\subsubsection{Equações e fórmulas}
% ----------------------------------------------------------

As equações e fórmulas devem ser destacadas no texto para facilitar a leitura.  Para numerá-las, usar algarismos arábicos entre parênteses e alinhados à direita. Pode-se adotar uma entrelinha maior do que a usada no texto \cite{NBR14724:2011}.

Exemplos, \autoref{eq:Eq_1} e \autoref{eq:Eq_2}.

\begin{equation}\label{eq:Eq_1}
\gls{C} = 2 \gls{pi} \gls{r}
\end{equation}

\begin{equation}\label{eq:Eq_2}
\gls{A} = \gls{pi} \gls{r}^2
\end{equation}

% ----------------------------------------------------------
\subsubsubsection{Exemplo tabela}
% ----------------------------------------------------------

De acordo com \textcite{ibge1993}, tabela é uma forma não discursiva de apresentar informações em que os números representam a informação central. Ver \autoref{tab:Tab_1}.

\begin{table}[htb]
	\ABNTEXfontereduzida
	\caption{\label{tab:Tab_1}Médias concentrações urbanas 2010-2011.}
	\begin{tabular}{@{}p{3.0cm}p{1.5cm}p{2cm}p{2.5cm}p{2.5cm}p{2.5cm}@{}}
		\toprule
		\textbf{Média concentração urbana} & \multicolumn{2}{l}{\textbf{População}} & \textbf{Produto Interno Bruto – PIB (bilhões R\$)} & \textbf{Número de empresas} & \textbf{Número de unidades locais} \\ \midrule
		\textbf{Nome}                      & \textbf{Total}   & \textbf{No Brasil}  &                                                   &                             & \\
		                      &    &    &                                                   &                             & \\
		Ji-Paraná (RO)                     & 116 610          & 116 610             & 1,686                                             & 2 734                       & 3 082 \\
		Parintins (AM)                     & 102 033          & 102 033             & 0,675                                             & 634                         & 683 \\
		Boa Vista (RR)                     & 298 215          & 298 215             & 4,823                                             & 4 852                       & 5 187 \\
		Bragança (PA)                      & 113 227          & 113 227             & 0,452                                             & 654                         & 686 \\ \bottomrule
	\end{tabular}
	\fonte{\textcite{ibge2016}.}
\end{table}