\chapter{Lista de Abreviaturas nos Artefatos do Documento}
Lista de Abreviaturas nos Artefatos do Documento
% ----------------------------------------------------------
\begin{table}[htb]
	\centering
	\caption{\label{Formatação do texto.}Requisitos funcionais}	
	\begin{tabular}{|p{4cm}|m{3cm}|p{7cm}|}
		\hline
		\textbf{Artefato} & \textbf{Termo ou sigla} & \textbf{Significado} \\ \hline
		Lista de Requisitos & RF & Requisito funcional: Funcionalidades que um sistema deve possuir. \\ \hline
		Lista de Requisitos & RNF & Requisito não funcional: Características que o software deve possuir, não apresenta funcionalidades. \\ \hline
		Lista de Regras de Negócio & RN & Regra de Negócio: Conjunto de diretizes que orientam como deve ser o funionamente das funcionalidades. \\ \hline
	\end{tabular}
	\fonte{\textcite{Elaborado pelos autores(2023)}.}
\end{table}
% ----------------------------------------------------------
\chapter{Introdução}
% ----------------------------------------------------------


% ----------------------------------------------------------
\section{Justificativa e Delimitação do Tema}
% ----------------------------------------------------------

A prestação de serviços gerais é uma atividade fundamental para a sociedade, oferecendo serviços variados como limpeza, conservação, segurança, jardinagem, manutenção predial, entre outros. Esses serviços são executados por empresas ou profissionais autônomos com o objetivo de atender às demandas de diferentes setores do mercado, tais como residências, condomínios, empresas e instituições públicas e privadas.

Porém, o setor enfrenta desafios significativos, como a identificação do perfil do público-alvo, a avaliação das principais necessidades e demandas do mercado, a compreensão dos aspectos jurídicos e regulatórios, a gestão eficiente de contratos e equipes e a busca pela qualidade na prestação de serviços. Além disso, a pandemia da COVID-19 trouxe ainda mais desafios, como a adaptação às novas medidas de segurança e a necessidade de se reinventar para atender às novas demandas do mercado.

Nesse contexto, a plataforma Empregai surge como uma solução inovadora, que tem como objetivo conectar trabalhadores a potenciais clientes que estejam buscando por serviços em sua região. Através dessa plataforma, trabalhadores podem encontrar novas oportunidades de trabalho em sua área de atuação e clientes podem encontrar profissionais qualificados para realizar serviços diversos.

Com a disponibilidade da plataforma em dispositivos móveis, a ferramenta oferece praticidade e eficiência para ambas as partes. A aplicação Empregai é uma excelente alternativa para os trabalhadores que buscam alternativas para garantir uma forma de subsistência ou complementar a renda, e para os clientes que procuram serviços de qualidade com profissionais qualificados.

% ----------------------------------------------------------
\section{Problemática}
% ----------------------------------------------------------

No contexto atual, é comum a busca por serviços gerais, entre outros. No entanto, de acordo com \textcite{FEUP2009} muitas vezes, a contratação desses serviços pode se tornar um desafio, especialmente quando se trata da confiança nos prestadores. A falta de informações precisas sobre os profissionais disponíveis, bem como a ausência de referências confiáveis, pode levar a escolhas equivocadas, prejuízos financeiros e até mesmo a situações de risco.

O mercado de serviços gerais é bastante amplo, mas a falta de confiança nos prestadores é uma questão relevante que precisa ser abordada. A ausência de referências precisas e confiáveis dos profissionais, bem como a dificuldade em encontrar opções com boa reputação, dificultam a tomada de decisão dos consumidores.

% ----------------------------------------------------------
\section{Objetivos}
% ----------------------------------------------------------



% ----------------------------------------------------------
\subsection{Objetivo Geral}
% ----------------------------------------------------------
O objetivo deste trabalho é desenvolver duas plataformas distintas que conectem de forma eficiente prestadores de serviços e clientes, visando proporcionar benefícios para ambas as partes.
% ----------------------------------------------------------
\subsection{Objetivos Específicos}
\begin{itemize}
  \item Realizar uma pesquisa de mercado.
  \item Realizar o levantamento de requisistos.
  \item Seguir o modelo incremental de desenvolvimento.
  \item Utilizar o Expo como plataforma de desenvolvimento móvel para a criação da aplicação que conectará prestadores de serviço e clientes.
  \item Utilizar o framework Gin para desenvolver a API RESTful da plataforma.
  \item Realizar testes de desempenho das plataformas, 
  \item Realizar prototipagem das telas utilizando o Figma.
\end{itemize}
% ----------------------------------------------------------
