% ----------------------------------------------------------
\chapter{Gerência do Projeto}\label{cap:desenvolvimento}
% ----------------------------------------------------------
Deve-se inserir texto entre as seções.
% ----------------------------------------------------------
\subsection{Cronograma}

Definimos o cronograma de atividades para desenvolvimento do trabalho de conclusão de curso conforme listado abaixo, e o cronograma de execução e previsões de conclusão na Tabela \ref{tab:cronograma}.

\begin{itemize}
    \item \textbf{Atividade 01} - Termo de aceite.
    \item \textbf{Atividade 02} - Definição do escopo do projeto.
    \item \textbf{Atividade 03} - Referencial Teórico.
    \item \textbf{Atividade 04} - Processo de desenvolvimento de Software.
    \item \textbf{Atividade 05} - Levantamento de requisitos.
    \item \textbf{Atividade 06} - Preparação dos slides.
    \item \textbf{Atividade 07} - Realização da banca.
    \item \textbf{Atividade 08} - Correções da banca.
    \item \textbf{Atividade 09} - Prototipação dos requisistos RF01 e RF02.
    \item \textbf{Atividade 10} - Implementação dos casos de usos Manter usuário e Manter Login.
    \item \textbf{Atividade 11} - Prototipação dos requisistos RF03 e RF07.
    \item \textbf{Atividade 12} - Implementação dos casos de usos Cadastro de anúncio e Candidatar ao anúncio.
    \item \textbf{Atividade 13} - Prototipação do requisisto RF08.
    \item \textbf{Atividade 14} - Implementação do caso de uso Concluir anúncio.
    \item \textbf{Atividade 15} - Prototipação do requisisto RF04.
    \item \textbf{Atividade 16} - Implementação do caso de uso Avaliação do profissional.
    \item \textbf{Atividade 17} - Prototipação dos requisistos RF05 e RF06.
    \item \textbf{Atividade 18} - Implementação do Histórico
\end{itemize}

\begin{table*}[ht]
\centering
\caption{Cronograma das atividades}
\label{tab:cronograma}
% \begin{tabular}{llllllllllllllllllll}
    \begin{tabular}{|c|c|c|c|c|c|c|c|c|c|c|c|c|}
        \hline & \multicolumn{11}{|c|}{2023} & \multicolumn{1}{|c|}{} \\
        \hline \multicolumn{1}{|c|}{Atv} & F & M & A & M & J & J & A & S & O & N & D & {Responsável} \\
        \hline \textbf{01} & & X& & & & & & & & & & Termo de aceite \\
        \hline \textbf{02} & & X& X& X& & & & & & & & Escopo do projeto \\
        \hline \textbf{03} & & & & X& X& & & & & & & Referencial Teórico\\
        \hline \textbf{04} & & & & X& X& & & & & & & Metodologia\\
        \hline \textbf{05} & & & & X& X& & & & & & & Levantamento de requisitos\\
        \hline \textbf{06} & & & & & X& & & & & & & Preparação dos slides\\
        \hline \textbf{07} & & & & & X& & & & & & & Realização da banca\\
        \hline \textbf{08} & & & & & X& & & & & & & Correções da banca\\
        \hline \textbf{09} & & & & & X& & & & & & & Prototipação 1\\
        \hline \textbf{10} & & & & & & X& & & & & & CS Manter usuário\\
        \hline \textbf{11} & & & & & & X& & & & & & Prototipação 2\\
        \hline \textbf{12} & & & & & & & X& & & & & CS Cadastro de anúncio\\
        \hline \textbf{13} & & & & & & & X& & & & & Prototipação 3\\
        \hline \textbf{14} & & & & & & & & X& & & & CS Concluir anúncio\\
        \hline \textbf{15} & & & & & & & & X& & & & Prototipação 4\\
        \hline \textbf{16} & & & & & & & & X& & & & CS Avaliação do profissional\\
        \hline \textbf{17} & & & & & & & & X& & & & Prototipação 5\\
        \hline \textbf{18} & & & & & & & & & X& & & CS Histórico\\
        \hline
    \end{tabular} 
\end{table*}

    
    
    