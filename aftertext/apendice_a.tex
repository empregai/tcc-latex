% % ----------------------------------------------------------
% \chapter{Descrição}
% % ----------------------------------------------------------

% Textos elaborados pelo autor, a fim de completar a sua argumentação. Deve ser precedido da palavra APÊNDICE, identificada por letras maiúsculas consecutivas, travessão e pelo respectivo título. Utilizam-se letras maiúsculas dobradas quando esgotadas as letras do alfabeto. 

% \begin{quadro}[htb]
% 	\centering
% 	\caption{\label{qua:Quadro_2}Modelo A.}	
% \begin{tabular}{|l|l|}
% \hline
% xxxx              & yyyyyyyyyyyyyyy    \\
% \hline
% xxxx              & yyyyyyyyyyyyyyy    \\
% \hline
% xxxx              & yyyyyyyyyyyyyyy    \\
% \hline
% xxxx              & yyyyyyyyyyyyyyy    \\
% \hline
% xxxx              & yyyyyyyyyyyyyyy    \\
% \hline
% xxxx              & yyyyyyyyyyyyyyy    \\
% \hline
% xxxx              & yyyyyyyyyyyyyyy    \\
% \hline
% rrrrrrrrrrrrrrrrr & eeeeeeeeeeeeeeeee  \\
% \hline
% xxxx              & yyyyyyyyyyyyyyy    \\
% \hline
% xxxx              & yyyyyyyyyyyyyyy    \\
% \hline
% rrrrrrrrrrrrrrrrr & eeeeeeeeeeeeeeeee  \\
% \hline
% xxxx              & yyyyyyyyyyyyyyy    \\
% \hline
%                   & ttttttttttttttttt  \\
% \hline
% rrrrrrrrrrrrrrrrr & eeeeeeeeeeeeeeeee  \\
% \hline
% ttttttttttttt     &                    \\
% \hline
% rrrrrrrrrrrrrrrrr & eeeeeeeeeeeeeeeee  \\
% \hline
% rrrrrrrrrrrrrrrrr & eeeeeeeeeeeeeeeee  \\
% \hline
%                   & gggggggggggggggggg \\
% \hline
% rrrrrrrrrrrrrrrrr & eeeeeeeeeeeeeeeee  \\
% \hline
% rrrrrrrrrrrrrrrrr & eeeeeeeeeeeeeeeee  \\
% \hline
% rrrrrrrrrrrrrrrrr & eeeeeeeeeeeeeeeee  \\
% \hline
% rrrrrrrrrrrrrrrrr & eeeeeeeeeeeeeeeee  \\
% \hline
% \end{tabular}
% \fonte{Elaborada pelo autor (2016).}
% \end{quadro}