% ---
% Capa
% ---
\imprimircapa
% ---

% ---
% Folha de rosto
% (o * indica que haverá a ficha bibliográfica)
% ---
\imprimirfolhaderosto*
% ---


% ---
% Inserir folha de aprovação
% ---
\begin{folhadeaprovacao}
	\OnehalfSpacing
	\centering
	\imprimirautor\\%
	\vspace*{10pt}		
	\textbf{\imprimirtitulo}%
	\ifnotempty{\imprimirsubtitulo}{:~\imprimirsubtitulo}\\%
	%		\vspace*{31.5pt}%3\baselineskip
	\vspace*{\baselineskip}
	%\begin{minipage}{\textwidth}
	% ~do~\imprimirprograma~do~\imprimircentro~da~\imprimirinstituicao~para~a~obtenção~do~título~de~\imprimirformacao.
	Este~\imprimirtipotrabalho~foi julgado adequado para obtenção do Título de “\imprimirformacao” e aprovado em sua forma final pelo~\imprimirprograma. \\
		\vspace*{\baselineskip}
	\imprimirlocal, \imprimirdata. \\
	\vspace*{2\baselineskip}
	\assinatura{\OnehalfSpacing\imprimircoordenador \\ \imprimircoordenadorRotulo~do Curso}
	\vspace*{2\baselineskip}
	\textbf{Banca Examinadora:} \\
	\vspace*{\baselineskip}
	\assinatura{\OnehalfSpacing\imprimirorientador \\ \imprimirorientadorRotulo}
	%\end{minipage}%
	\vspace*{\baselineskip}
	\assinatura{Prof.(a) xxxx, Dr(a).\\
	Avaliador(a) \\
	Instituição xxxx}

	\vspace*{\baselineskip}
	\assinatura{Prof.(a) xxxx, Dr(a).\\
	Avaliador(a) \\
	Instituição xxxx}


\end{folhadeaprovacao}
% ---

% ---
% Dedicatória
% ---
\begin{dedicatoria}
	\vspace*{\fill}
	\noindent
	\begin{adjustwidth*}{}{5.5cm}     
		Este trabalho é dedicado aos meus colegas de classe e aos meus queridos pais.
	\end{adjustwidth*}
\end{dedicatoria}
% ---

% ---
% Agradecimentos
% ---
\begin{agradecimentos}
	Inserir os agradecimentos aos colaboradores à execução do trabalho. 
	
	Xxxxxxxxxxxxxxxxxxxxxxxxxxxxxxxxxxxxxxxxxxxxxxxxxxxxxxxxxxxxxxxxxxxxxx. 
\end{agradecimentos}
% ---

% ---
% Epígrafe
% ---
\begin{epigrafe}
	\vspace*{\fill}
	\begin{flushright}
		\textit{``Texto da Epígrafe.\\
			Citação relativa ao tema do trabalho.\\
			É opcional. A epígrafe pode também aparecer\\
			na abertura de cada seção ou capítulo.\\
			Deve ser elaborada de acordo com a NBR 10520.''\\
			(Autor da epígrafe, ano)}
	\end{flushright}
\end{epigrafe}
% ---

% ---
% RESUMOS
% ---

% resumo em português
\setlength{\absparsep}{18pt} % ajusta o espaçamento dos parágrafos do resumo
\begin{resumo}
	\SingleSpacing
	Este trabalho aborda o desenvolvimento da plataforma Empregai, cujo objetivo é facilitar a conexão entre prestadores de serviços e clientes. Com o aumento do 
	uso de smartphones devido à pandemia de COVID-19, vários serviços tiveram que se adaptar a essa nova realidade. Nesse cenário, muitas pessoas estão procurando 
	profissionais capacitados e bem avaliados para contratar.

    A plataforma Empregai surge como uma solução ao conectar trabalhadores a potenciais clientes por meio de um aplicativo móvel. Essa ferramenta oferece praticidade
	e eficiência, permitindo que os trabalhadores encontrem oportunidades de trabalho em suas áreas de atuação, enquanto os clientes encontram profissionais qualificados
	para realizar diversos serviços.

	\textbf{Palavras-chave}: Desenvolvimento. Incremental. Prestador de serviços.
\end{resumo}

% resumo em inglês
\begin{resumo}[Abstract]
	\SingleSpacing
	\begin{otherlanguage*}{english}
		Resumo traduzido para outros idiomas, neste caso, inglês. Segue o formato do resumo feito na língua vernácula. As palavras-chave traduzidas, versão em língua estrangeira, são colocadas abaixo do texto precedidas pela expressão “Keywords”, separadas por ponto.
		
		\textbf{Keywords}: Keyword 1. Keyword 2. Keyword 3.
	\end{otherlanguage*}
\end{resumo}

%% resumo em francês 
%\begin{resumo}[Résumé]
% \begin{otherlanguage*}{french}
%    Il s'agit d'un résumé en français.
% 
%   \textbf{Mots-clés}: latex. abntex. publication de textes.
% \end{otherlanguage*}
%\end{resumo}
%
%% resumo em espanhol
%\begin{resumo}[Resumen]
% \begin{otherlanguage*}{spanish}
%   Este es el resumen en español.
%  
%   \textbf{Palabras clave}: latex. abntex. publicación de textos.
% \end{otherlanguage*}
%\end{resumo}
%% ---

{%hidelinks
	\hypersetup{hidelinks}
	% ---
	% inserir lista de ilustrações
	% ---
	\pdfbookmark[0]{\listfigurename}{lof}
	\listoffigures*
	\cleardoublepage
	% ---
	
	% ---
	% inserir lista de quadros
	% ---
	\pdfbookmark[0]{\listofquadrosname}{loq}
	\listofquadros*
	\cleardoublepage
	% ---
	
	% ---
	% inserir lista de tabelas
	% ---
	\pdfbookmark[0]{\listtablename}{lot}
	\listoftables*
	\cleardoublepage
	% ---
	
	% ---
	% inserir lista de abreviaturas e siglas (devem ser declarados no preambulo)
	% ---
	\imprimirlistadesiglas
	% ---
	
	% ---
	% inserir lista de símbolos (devem ser declarados no preambulo)
	% ---
	\imprimirlistadesimbolos
	% ---
	
	% ---
	% inserir o sumario
	% ---
	\pdfbookmark[0]{\contentsname}{toc}
	\tableofcontents*
	\cleardoublepage
	
}%hidelinks
% ---