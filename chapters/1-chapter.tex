% ----------------------------------------------------------
\chapter{Introdução}
% ----------------------------------------------------------

Inserir aqui o texto da introdução

jksdbfkjbhkdsjhskjhdkjhsdkjhsdkjfh

% ----------------------------------------------------------
\section{Justificativa e Delimitação do Tema}
% ----------------------------------------------------------

Inserir aqui o texto da justificativa do trabalho e a delimitação do tema

% ----------------------------------------------------------
\section{Problemática}
% ----------------------------------------------------------

Inserir aqui o texto da problemática

% ----------------------------------------------------------
\section{Objetivos}
% ----------------------------------------------------------



% ----------------------------------------------------------
\subsection{Objetivo Geral}
% ----------------------------------------------------------
O objetivo geral do sistema é fornecer uma plataforma que facilite a conexão entre prestadores de serviço e clientes, 
criando benefícios para ambas as partes. Os prestadores de serviço terão acesso a mais oportunidades de trabalho, 
enquanto os clientes poderão encontrar profissionais qualificados para atender às suas necessidades.

% ----------------------------------------------------------
\subsection{Objetivos Específicos}
% ----------------------------------------------------------

Além disso, o sistema tem como objetivo específico desenvolver dois aplicativos móveis utilizando a tecnologia React Native: um para os prestadores de serviço e outro para os clientes. O aplicativo permitirá que os clientes anunciem seus serviços e que os prestadores de serviço se candidatem a essas oportunidades. O sistema também terá uma função de avaliação para permitir que os clientes possam avaliar o serviço prestado pelos prestadores.

% ----------------------------------------------------------
\section{Cronograma}
% ----------------------------------------------------------


Definimos o cronograma de atividades para desenvolvimento do trabalho de conclusão de curso conforme listado abaixo, e o cronograma de execução e previsões de conclusão na Tabela \ref{tab:cronograma}.

\begin{itemize}
    \item \textbf{Atividade 01} - Definição do tema.
    
    \item \textbf{Atividade 02} - Definição do escopo.
   
    \item \textbf{Atividade 03} - Levantamento de requisitos.
    
    \item \textbf{Atividade 04} - Definição do MVP.
    
    \item \textbf{Atividade 05} - Introdução.
    
    \item \textbf{Atividade 06} - Definição e escrita da metodologia a ser utilizada para desenvolvimento do trabalho, resultados alcançados e  resultados esperados.
    
    \item \textbf{Atividade 07} - Montagem do protótipo experimental do projeto
    
    \item \textbf{Atividade 08} - Coleta de dados/resultados da implementação do projeto
    
    \item \textbf{Atividade 09} - Escrita da Análise e Discussão dos Resultados
    
    \item \textbf{Atividade 10} - Escrita do texto da monografia
    
    \item \textbf{Atividade 11} - Revisão do texto da monografia e escrita das considerações finais
    
    \item \textbf{Atividade 12} - Entrega da monografia aos avaliadores
    
    \item \textbf{Atividade 13} - Defesa da monografia
    
\end{itemize}

\begin{table*}[ht]
\centering
\caption{Cronograma das atividades}
\label{tab:cronograma}
% \begin{tabular}{llllllllllllllllllll}
\begin{tabular}{|c|c|c|c|c|c|c|c|c|c|c|c|c|}
\hline & \multicolumn{11}{|c|}{2023} & \multicolumn{1}{|c|}{} \\
\hline \multicolumn{1}{|c|}{Atv} & F & M & A & M & J & J & A & S & O & N & D & {Responsável} \\
\hline \textbf{01} & & X& & & & & & & & & & Orientador/Orientandos \\
\hline \textbf{02} & & X& & & & & & & & & & Orientandos \\
\hline \textbf{03} & & X& & & & & & & & & & Orientandos\\
\hline \textbf{04} & & X& & & & & & & & & & Orientandos \\
\hline \textbf{05} & & X& & & & & & & & & & Orientandos \\
\hline \textbf{06} & & & & & X& & & & & & & Orientandos \\
\hline \textbf{07} & & & & & X& X& X& X& & & & Orientandos \\
\hline \textbf{08} & & & & & & & X& X& & & & Orientandos \\
\hline \textbf{09} & & & & & & & & X& & & & Orientandos \\
\hline \textbf{10} & & & & & & & & X& X & X & & Orientandos \\
\hline \textbf{11} & & & & & & & & & & X & & Orientador/Orientandos \\
\hline \textbf{12} & & & & & & & & & & & X & Orientandos \\
\hline \textbf{13} & & & & & & & & & & & X & Orientandos \\
\hline
\end{tabular} 
\end{table*}


