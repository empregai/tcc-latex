% ----------------------------------------------------------
\chapter{Introdução}
% ----------------------------------------------------------


% ----------------------------------------------------------
\section{Justificativa e Delimitação do Tema}
% ----------------------------------------------------------

A prestação de serviços gerais é uma atividade fundamental para a sociedade, oferecendo serviços variados como limpeza, conservação, segurança, jardinagem, manutenção predial, entre outros. Esses serviços são executados por empresas ou profissionais autônomos com o objetivo de atender às demandas de diferentes setores do mercado, tais como residências, condomínios, empresas e instituições públicas e privadas.

Porém, o setor enfrenta desafios significativos, como a identificação do perfil do público-alvo, a avaliação das principais necessidades e demandas do mercado, a compreensão dos aspectos jurídicos e regulatórios, a gestão eficiente de contratos e equipes e a busca pela qualidade na prestação de serviços. Além disso, a pandemia da COVID-19 trouxe ainda mais desafios, como a adaptação às novas medidas de segurança e a necessidade de se reinventar para atender às novas demandas do mercado.

Nesse contexto, a plataforma Empregai surge como uma solução inovadora, que tem como objetivo conectar trabalhadores a potenciais clientes que estejam buscando por serviços em sua região. Através dessa plataforma, trabalhadores podem encontrar novas oportunidades de trabalho em sua área de atuação e clientes podem encontrar profissionais qualificados para realizar serviços diversos.

Com a disponibilidade da plataforma em dispositivos móveis, a ferramenta oferece praticidade e eficiência para ambas as partes. A aplicação Empregai é uma excelente alternativa para os trabalhadores que buscam alternativas para garantir uma forma de subsistência ou complementar a renda, e para os clientes que procuram serviços de qualidade com profissionais qualificados.

% ----------------------------------------------------------
\section{Problemática}
% ----------------------------------------------------------

No contexto atual, é comum a busca por serviços gerais, entre outros. No entanto, de acordo com \textcite{FEUP2009} muitas vezes, a contratação desses serviços pode se tornar um desafio, especialmente quando se trata da confiança nos prestadores. A falta de informações precisas sobre os profissionais disponíveis, bem como a ausência de referências confiáveis, pode levar a escolhas equivocadas, prejuízos financeiros e até mesmo a situações de risco.

O mercado de serviços gerais é bastante amplo, mas a falta de confiança nos prestadores é uma questão relevante que precisa ser abordada. A ausência de referências precisas e confiáveis dos profissionais, bem como a dificuldade em encontrar opções com boa reputação, dificultam a tomada de decisão dos consumidores.

% ----------------------------------------------------------
\section{Objetivos}
% ----------------------------------------------------------



% ----------------------------------------------------------
\subsection{Objetivo Geral}
% ----------------------------------------------------------
O objetivo geral do sistema é fornecer uma plataforma que facilite a conexão entre prestadores de serviço e clientes, 
criando benefícios para ambas as partes. Os prestadores de serviço terão acesso a mais oportunidades de trabalho, 
enquanto os clientes poderão encontrar profissionais qualificados para atender às suas necessidades.

% ----------------------------------------------------------
\subsection{Objetivos Específicos}
% ----------------------------------------------------------

Além disso, o sistema tem como objetivo específico desenvolver dois aplicativos móveis utilizando a tecnologia React Native: um para os prestadores de serviço e outro para os clientes. O aplicativo permitirá que os clientes anunciem seus serviços e que os prestadores de serviço se candidatem a essas oportunidades. O sistema também terá uma função de avaliação para permitir que os clientes possam avaliar o serviço prestado pelos prestadores.

% ----------------------------------------------------------
\section{Cronograma}
% ----------------------------------------------------------


Definimos o cronograma de atividades para desenvolvimento do trabalho de conclusão de curso conforme listado abaixo, e o cronograma de execução e previsões de conclusão na Tabela \ref{tab:cronograma}.

\begin{itemize}
    \item \textbf{Atividade 01} - Termo de aceite.
    \item \textbf{Atividade 02} - Definição do escopo do projeto.
    \item \textbf{Atividade 03} - Processo de desenvolvimento de Software.
    \item \textbf{Atividade 04} - Propotipação.
    \item \textbf{Atividade 05} - Preparação dos slides.
    \item \textbf{Atividade 06} - Realização da banca.
    \item \textbf{Atividade 07} - Correções da banca.
\end{itemize}

\begin{table*}[ht]
\centering
\caption{Cronograma das atividades}
\label{tab:cronograma}
% \begin{tabular}{llllllllllllllllllll}
\begin{tabular}{|c|c|c|c|c|c|c|c|c|c|c|c|c|}
\hline & \multicolumn{11}{|c|}{2023} & \multicolumn{1}{|c|}{} \\
\hline \multicolumn{1}{|c|}{Atv} & F & M & A & M & J & J & A & S & O & N & D & {Responsável} \\
\hline \textbf{01} & & X& & & & & & & & & & Orientador/Orientandos \\
\hline \textbf{02} & & X& X& X& & & & & & & & Orientandos \\
\hline \textbf{03} & & & & & & & & & & & & Orientandos\\
\hline \textbf{04} & & & & & & & & & & & & Orientandos\\
\hline \textbf{05} & & & & & & & & & & & & Orientandos\\
\hline \textbf{06} & & & & & & & & & & & & Orientandos\\
\hline \textbf{07} & & & & & & & & & & & & Orientandos\\
\hline
\end{tabular} 
\end{table*}


