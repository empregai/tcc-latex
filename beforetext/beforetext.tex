% ---
% Capa
% ---
\imprimircapa
% ---

% ---
% Folha de rosto
% (o * indica que haverá a ficha bibliográfica)
% ---
\imprimirfolhaderosto*
% ---


% ---
% Inserir folha de aprovação
% ---
\begin{folhadeaprovacao}
	\OnehalfSpacing
	\centering
	\imprimirautor\\%
	\vspace*{10pt}		
	\textbf{\imprimirtitulo}%
	\ifnotempty{\imprimirsubtitulo}{:~\imprimirsubtitulo}\\%
	%		\vspace*{31.5pt}%3\baselineskip
	\vspace*{\baselineskip}
	%\begin{minipage}{\textwidth}
	% ~do~\imprimirprograma~do~\imprimircentro~da~\imprimirinstituicao~para~a~obtenção~do~título~de~\imprimirformacao.
	Este~\imprimirtipotrabalho~foi julgado adequado para obtenção do Título de “\imprimirformacao” e aprovado em sua forma final pelo~\imprimirprograma. \\
		\vspace*{\baselineskip}
	\imprimirlocal, \imprimirdata. \\
	\vspace*{2\baselineskip}
	\assinatura{\OnehalfSpacing\imprimircoordenador \\ \imprimircoordenadorRotulo~do Curso}
	\vspace*{2\baselineskip}
	\textbf{Banca Examinadora:} \\
	\vspace*{\baselineskip}
	\assinatura{\OnehalfSpacing\imprimirorientador \\ \imprimirorientadorRotulo}
	%\end{minipage}%
	\vspace*{\baselineskip}
	\assinatura{Prof.(a) xxxx, Dr(a).\\
	Avaliador(a) \\
	Instituição xxxx}

	\vspace*{\baselineskip}
	\assinatura{Prof.(a) xxxx, Dr(a).\\
	Avaliador(a) \\
	Instituição xxxx}


\end{folhadeaprovacao}
% ---

% ---
% Dedicatória
% ---
\begin{dedicatoria}
	\vspace*{\fill}
	\noindent
	\begin{adjustwidth*}{}{5.5cm}     
		Este trabalho é dedicado aos meus colegas de classe e aos meus queridos pais.
	\end{adjustwidth*}
\end{dedicatoria}
% ---

% ---
% Agradecimentos
% ---
\begin{agradecimentos}
	Gostaria de expressar minha sincera gratidão a todos os membros do meu grupo de trabalho no Trabalho de Conclusão de Curso (TCC). Juntos, enfrentamos desafios e superamos obstáculos, e essa conquista não seria possível sem a colaboração e dedicação de cada um de vocês. 

	Agradeço também a nosso orientador William Pereira dos Santos Junior por sua orientação e apoio durante todo o processo. Sua expertise e visão crítica foram essenciais para o aprimoramento de nosso trabalho. Valorizo profundamente seu compromisso conosco e sua dedicação em nos guiar na direção certa.

	Por fim, quero expressar minha gratidão a todas as pessoas que estiveram ao nosso lado, como familiares e amigos, durante essa jornada desafiadora. Seu apoio, incentivo e compreensão foram essenciais para nos manter motivados e confiantes em nosso trabalho.

	Mais uma vez, meu sincero agradecimento a cada membro do grupo e a todas as pessoas envolvidas nessa jornada do TCC em grupo. Juntos, alcançamos um resultado que nos enche de orgulho e satisfação. Obrigado(a) a todos por tornarem essa experiência tão enriquecedora e memorável.
	
\end{agradecimentos}
% ---

% ---
% Epígrafe
% ---
\begin{epigrafe}
	\vspace*{\fill}
	\begin{flushright}
		\textit{``Texto da Epígrafe.\\
			Citação relativa ao tema do trabalho.\\
			É opcional. A epígrafe pode também aparecer\\
			na abertura de cada seção ou capítulo.\\
			Deve ser elaborada de acordo com a NBR 10520.''\\
			(Autor da epígrafe, ano)}
	\end{flushright}
\end{epigrafe}
% ---

% ---
% RESUMOS
% ---

% resumo em português
\setlength{\absparsep}{18pt} % ajusta o espaçamento dos parágrafos do resumo
\begin{resumo}
	\SingleSpacing
Este trabalho aborda o desenvolvimento da plataforma Empregai, um aplicativo móvel projetado para facilitar a contratação e a prestação de serviços gerais, conectando prestadores de serviços e clientes. Com o aumento do uso de smartphones, impulsionado pela pandemia de COVID-19, muitos serviços precisaram se adaptar a essa nova realidade. Nesse contexto, há uma demanda crescente por profissionais qualificados e bem avaliados.

A plataforma Empregai surge como uma solução ao oferecer uma interface intuitiva e eficiente, permitindo que os trabalhadores encontrem oportunidades de trabalho em suas respectivas áreas de atuação e que os clientes encontrem profissionais qualificados para realizar diversos tipos de serviços. Através do aplicativo móvel, os usuários podem criar perfis, fornecer informações sobre suas habilidades e experiências, além de listar os serviços que estão dispostos a oferecer.

Com recursos avançados de busca e filtragem, os clientes podem encontrar profissionais adequados às suas necessidades específicas, levando em consideração avaliações e comentários de outros usuários. Além disso, o aplicativo permite o agendamento e a comunicação direta entre prestadores de serviços e clientes, tornando o processo de contratação mais ágil e conveniente para ambas as partes.

Ao promover a conexão direta entre trabalhadores e clientes, a plataforma Empregai busca otimizar o processo de contratação de serviços gerais, eliminando intermediários e reduzindo custos. Além disso, visa fornecer um ambiente seguro e confiável, onde os usuários possam encontrar profissionais qualificados e confiáveis para atender às suas necessidades.

Através deste trabalho, espera-se contribuir para a melhoria do mercado de contratação e prestação de serviços gerais, fornecendo uma solução tecnológica inovadora que beneficie tanto prestadores de serviços quanto clientes. A plataforma Empregai representa uma oportunidade para facilitar a conexão entre profissionais e clientes, promovendo a eficiência e a praticidade na contratação de serviços gerais.

	\textbf{Palavras-chave}: Melhoria do mercado de contratação. Aplicativo móvel. Ambiente seguro e confiável. Avaliação de profissionais. Contratação de serviços. Prestador de serviços.
\end{resumo}

% resumo em inglês
\begin{resumo}[Abstract]
	\SingleSpacing
	\begin{otherlanguage*}{english}
This work addresses the development of the Empregai platform, aimed at facilitating the connection between service providers and clients. With the increased use of smartphones due to the COVID-19 pandemic, various services have had to adapt to this new reality. In this scenario, many people are looking for skilled and well-reviewed professionals to hire.

The Empregai platform emerges as a solution by connecting workers to potential clients through a mobile application. This tool offers convenience and efficiency, allowing workers to find job opportunities in their respective fields, while clients find qualified professionals to carry out various services.

This work addresses the development of the Empregai platform, aimed at facilitating the connection between service providers and clients. With the increased use of smartphones due to the COVID-19 pandemic, various services have had to adapt to this new reality. In this scenario, many people are looking for skilled and well-reviewed professionals to hire.

The Empregai platform emerges as a solution by connecting workers to potential clients through a mobile application. This tool offers convenience and efficiency, allowing workers to find job opportunities in their respective fields, while clients find qualified professionals to carry out various services.

Through the mobile application, users can create profiles, provide information about their skills and experiences, and list the services they are willing to offer. With advanced search and filtering features, clients can find professionals suitable for their specific needs, taking into account ratings and reviews from other users. Additionally, the application allows for scheduling and direct communication between service providers and clients, making the hiring process more agile and convenient for both parties.

By promoting direct connection between workers and clients, the Empregai platform aims to optimize the process of hiring general services, eliminating intermediaries and reducing costs. Furthermore, it aims to provide a secure and trustworthy environment where users can find qualified and reliable professionals to meet their needs.

Through this work, we hope to contribute to the improvement of the market for hiring and providing general services, providing an innovative technological solution that benefits both service providers and clients. The Empregai platform represents an opportunity to facilitate the connection between professionals and clients, promoting efficiency and convenience in the hiring of general services.

		\textbf{Keywords}: Improvement of the hiring market. Mobile application. Safe and reliable environment. Professional evaluation. Service hiring. Service provider.
	\end{otherlanguage*}
\end{resumo}

%% resumo em francês 
%\begin{resumo}[Résumé]
% \begin{otherlanguage*}{french}
%    Il s'agit d'un résumé en français.
% 
%   \textbf{Mots-clés}: latex. abntex. publication de textes.
% \end{otherlanguage*}
%\end{resumo}
%
%% resumo em espanhol
%\begin{resumo}[Resumen]
% \begin{otherlanguage*}{spanish}
%   Este es el resumen en español.
%  
%   \textbf{Palabras clave}: latex. abntex. publicación de textos.
% \end{otherlanguage*}
%\end{resumo}
%% ---

{%hidelinks
	\hypersetup{hidelinks}
	% ---
	% inserir lista de ilustrações
	% ---
	\pdfbookmark[0]{\listfigurename}{lof}
	\listoffigures*
	\cleardoublepage
	% ---
	
	% ---
	% inserir lista de quadros
	% ---
	\pdfbookmark[0]{\listofquadrosname}{loq}
	\listofquadros*
	\cleardoublepage
	% ---
	
	% ---
	% inserir lista de tabelas
	% ---
	\pdfbookmark[0]{\listtablename}{lot}
	\listoftables*
	\cleardoublepage
	% ---
	
	% ---
	% inserir lista de abreviaturas e siglas (devem ser declarados no preambulo)
	% ---
	\imprimirlistadesiglas
	% ---
	
	% ---
	% inserir lista de símbolos (devem ser declarados no preambulo)
	% ---
	\imprimirlistadesimbolos
	% ---
	
	% ---
	% inserir o sumario
	% ---
	\pdfbookmark[0]{\contentsname}{toc}
	\tableofcontents*
	\cleardoublepage
	
}%hidelinks
% ---